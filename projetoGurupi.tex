\documentclass[
	% -- opções da classe memoir --
	12pt,				% tamanho da fonte
	openright,			% capítulos começam em pág ímpar (insere página vazia caso preciso)
	oneside,			% para impressão em recto e verso. Oposto a oneside
	a4paper,			% tamanho do papel. 
	% -- opções da classe abntex2 --
	%chapter=TITLE,		% títulos de capítulos convertidos em letras maiúsculas
	%section=TITLE,		% títulos de seções convertidos em letras maiúsculas
	%subsection=TITLE,	% títulos de subseções convertidos em letras maiúsculas
	%subsubsection=TITLE,% títulos de subsubseções convertidos em letras maiúsculas
	% -- opções do pacote babel --
	english,			% idioma adicional para hifenização
	french,				% idioma adicional para hifenização
	spanish,			% idioma adicional para hifenização
	brazil				% o último idioma é o principal do documento
	]{abntex2}

% ---
% Pacotes básicos 
% ---
%\usepackage[regular]{Chivo}
\usepackage{lmodern}			% Usa a fonte Latin Modern			
\usepackage[T1]{fontenc}		% Selecao de codigos de fonte.
\usepackage[utf8]{inputenc}		% Codificacao do documento (conversão automática dos acentos)
\usepackage{lastpage}			% Usado pela Ficha catalográfica
\usepackage{indentfirst}		% Indenta o primeiro parágrafo de cada seção.
\usepackage{color}				% Controle das cores
\usepackage{graphicx}			% Inclusão de gráficos
\usepackage{microtype} 			% para melhorias de justificação
% ---
		
% ---
% Pacotes adicionais, usados apenas no âmbito do Modelo Canônico do abnteX2
% ---
\usepackage{lipsum}				% para geração de dummy text
% ---

% ---
% Pacotes de citações
% ---
%\usepackage[brazilian,hyperpageref]{backref}	 % Paginas com as citações na bibl
%\usepackage[alf]{abntex2cite}	% Citações padrão ABNT

% --- 
% CONFIGURAÇÕES DE PACOTES
% --- 


% ---
% Informações de dados para CAPA e FOLHA DE ROSTO
% ---
\titulo{Restauração de Imagens de Placas de Veículos Usando Super-resolução}
\autor{David Campos Anchieta}
\local{Belém - PA}
\data{2017}
\orientador{Prof. Dr. Ronaldo de Freitas Zampolo}
\instituicao{%
  Universidade Federal do Pará
  \par
  Instituto de Tecnologia
  \par
  Faculdade de Engenharia da computação e Telecomunicações}
\tipotrabalho{Trabalho de Conclusão de Curso}
% O preambulo deve conter o tipo do trabalho, o objetivo, 
% o nome da instituição e a área de concentração 

%ISSUE: redigir preambulo
\preambulo{ PREÂMBULO }
% ---


% ---
% Configurações de aparência do PDF final

% alterando o aspecto da cor azul
\definecolor{blue}{RGB}{41,5,195}

% informações do PDF
\makeatletter
\hypersetup{
     	%pagebackref=true,
		pdftitle={\@title}, 
		pdfauthor={\@author},
    	pdfsubject={\imprimirpreambulo},
	    pdfcreator={LaTeX with abnTeX2},
		pdfkeywords={abnt}{latex}{abntex}{abntex2}{trabalho acadêmico}, 
		colorlinks=true,       		% false: boxed links; true: colored links
    	linkcolor=blue,          	% color of internal links
    	citecolor=blue,        		% color of links to bibliography
    	filecolor=magenta,      		% color of file links
		urlcolor=blue,
		bookmarksdepth=4
}
\makeatother
% --- 

% --- 
% Espaçamentos entre linhas e parágrafos 
% --- 

% O tamanho do parágrafo é dado por:
\setlength{\parindent}{1.3cm}

% Controle do espaçamento entre um parágrafo e outro:
\setlength{\parskip}{0.2cm}  % tente também \onelineskip

% ---
% compila o indice
% ---
\makeindex
% ---

% ----
% Início do documento
% ----
\begin{document}

% Seleciona o idioma do documento (conforme pacotes do babel)
%\selectlanguage{english}
\selectlanguage{brazil}

% Retira espaço extra obsoleto entre as frases.
\frenchspacing 

% ----------------------------------------------------------
% ELEMENTOS PRÉ-TEXTUAIS
% ----------------------------------------------------------
% \pretextual

% ---
% Capa
% ---
\imprimircapa
% ---

% ---
% Folha de rosto
% (o * indica que haverá a ficha bibliográfica)
% ---
\imprimirfolhaderosto
% ---

% ---
% Inserir folha de aprovação
% ---

% Isto é um exemplo de Folha de aprovação, elemento obrigatório da NBR
% 14724/2011 (seção 4.2.1.3). Você pode utilizar este modelo até a aprovação
% do trabalho. Após isso, substitua todo o conteúdo deste arquivo por uma
% imagem da página assinada pela banca com o comando abaixo:
%
% \includepdf{folhadeaprovacao_final.pdf}
%

\begin{folhadeaprovacao}
%ISSUE: folha de aprovação
  \begin{center}
    {\ABNTEXchapterfont\large\imprimirautor}

    \vspace*{\fill}\vspace*{\fill}
    \begin{center}
      \ABNTEXchapterfont\bfseries\Large\imprimirtitulo
    \end{center}
    \vspace*{\fill}
    
    \hspace{.45\textwidth}
    \begin{minipage}{.5\textwidth}
        \imprimirpreambulo
    \end{minipage}%
    \vspace*{\fill}
   \end{center}
        
   Trabalho aprovado. \imprimirlocal, 24 de novembro de 2012:

   \assinatura{\textbf{\imprimirorientador} \\ Orientador} 
   \assinatura{\textbf{Professor} \\ Convidado 1}
   \assinatura{\textbf{Professor} \\ Convidado 2}
   %\assinatura{\textbf{Professor} \\ Convidado 3}
   %\assinatura{\textbf{Professor} \\ Convidado 4}
      
   \begin{center}
    \vspace*{0.5cm}
    {\large\imprimirlocal}
    \par
    {\large\imprimirdata}
    \vspace*{1cm}
  \end{center}
  
\end{folhadeaprovacao}
% ---

% ---
% Dedicatória
% ---
\begin{dedicatoria}
%ISSUE: dedicatória
   \vspace*{\fill}
   \centering
   \noindent
   \textit{ DEDICATÓRIA} \vspace*{\fill}
\end{dedicatoria}
% ---

% ---
% Agradecimentos
% ---
\begin{agradecimentos}
%ISSUE: agradecimentos


\end{agradecimentos}
% ---

% ---
% Epígrafe
% ---
\begin{epigrafe}
%ISSUE: epígrafe
    \vspace*{\fill}
	\begin{flushright}
		\textit{O que as suas mãos tiverem que fazer, que o façam com toda a sua força, pois na sepultura, para onde você vai, não há atividade nem planejamento, não há conhecimento nem sabedoria.\\
Eclesiastes 9:10}
	\end{flushright}
\end{epigrafe}
% ---

% ---
% RESUMOS
% ---

% resumo em português
\setlength{\absparsep}{18pt} % ajusta o espaçamento dos parágrafos do resumo
\begin{resumo}
%ISSUE: resumo pt

\end{resumo}

% resumo em inglês
\begin{resumo}[Abstract]
%ISSUE: resumo en
 \begin{otherlanguage*}{english}
   This is the english abstract.

   \vspace{\onelineskip}
 
   \noindent 
   \textbf{Keywords}: latex. abntex. text editoration.
 \end{otherlanguage*}
\end{resumo}


% ---
% inserir lista de ilustrações
% ---
\pdfbookmark[0]{\listfigurename}{lof}
\listoffigures*
\cleardoublepage
% ---

% ---
% inserir lista de tabelas
% ---
\pdfbookmark[0]{\listtablename}{lot}
\listoftables*
\cleardoublepage
% ---

% ---
% inserir lista de abreviaturas e siglas
% ---
\begin{siglas}
  \item[SR] Super-resolução
  \item[HR] \textit{High resolution} ou alta resolução
\end{siglas}
% ---

% ---
% inserir lista de símbolos
% ---
\begin{simbolos}
%ISSUE: lista de símbolos
  \item[$ \Gamma $] Letra grega Gama
  \item[$ \lambda $] Lambda
  \item[$ \zeta $] Letra grega minúscula zeta
  \item[$ \in $] Pertence
\end{simbolos}
% ---

% ---
% inserir o sumario
% ---
\pdfbookmark[0]{\contentsname}{toc}
\tableofcontents*
\cleardoublepage
% ---



% ----------------------------------------------------------
% ELEMENTOS TEXTUAIS
% ----------------------------------------------------------
\textual

% ----------------------------------------------------------
% Introdução (exemplo de capítulo sem numeração, mas presente no Sumário)
% ----------------------------------------------------------
\chapter[Introdução]{Introdução}
\addcontentsline{toc}{chapter}{Introdução}
% ----------------------------------------------------------
%ISSUE: introdução

\chapter{Restauração de Imagens Usando Super-resolução}
\section{Introdução}
A disponibilidade de um sistema de aquisição de imagens em alta resolução é um aspecto positivo para todas as aplicações que se beneficiam do uso de imagens digitais. Imagens médicas com boa resolução facilitam o diagnóstico de doenças, por exemplo \cite{park2003super}. No entanto, gerar imagens de alta resolução pode ter um custo elevado, além de apresentar outros inconvenientes dependendo da aplicação. Por exemplo, um sistema de vigilância que gerasse imagens de alta resolução precisaria de robustez computacional e alta capacidade de armazenamento para funcionar continuamente. Sistemas de imagens médicas também enfrentam o dilema de obter imagens melhores sem aumentar a exposição do paciente à radiação \cite{yue2016image}.

Todas essas questões trouxeram a necessidade de técnicas de aprimoramento de imagens que não somente aumentassem a resolução espacial artificialmente, como ocorre em um processo de interpolação, mas que também recuperassem as componentes de alta frequência que se perdem durante o processo de subamostragem.



Entende-se por Super-resolução o processo de obtenção de uma ou mais imagens de alta resolução a partir de uma ou mais imagens de baixa resolução. Técnicas de super-resolução são estudadas desde os anos 1970 têm despertado um grande interesse de estudo nas últimas décadas. As aplicações de técnicas de SR são diversas e incluem aprimoramentos de imagens médicas, de imagens de rosto, de texto e impressões digitais\cite{nasrollahi2014super}.

As abordagens também se diversificaram ao longo dos anos de estudos. Os primeiros algorítimos desenvolvidos faziam o processamento no domínio da frequência, usando transformações de Fourier e Wavelet. No entanto, os métodos de SR no domínio da frequência tinham limitações acabaram por ser substituídos por métodos no domínio espacial.



\section{Apreciação das Técnicas de Super-resolução}


\chapter{Super Resolução Bayesiana}
\section{Modelo de Observação}

\chapter{Metodologia}

\chapter{Resultados}

\chapter{Conclusão}

% ----------------------------------------------------------
% ELEMENTOS PÓS-TEXTUAIS
% ----------------------------------------------------------
\postextual
% ----------------------------------------------------------

% ----------------------------------------------------------
% Referências bibliográficas
% ----------------------------------------------------------
\bibliographystyle{plain}
\bibliography{referencias}


\end{document}
