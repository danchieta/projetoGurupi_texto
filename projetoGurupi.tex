\documentclass[12pt,openright,oneside,a4paper,english,brazil]{abntex2}

% Pacotes básicos 
\usepackage{lmodern}			% Usa a fonte Latin Modern			
\usepackage[T1]{fontenc}		% Selecao de codigos de fonte.
\usepackage[utf8]{inputenc}		% Codificacao do documento (conversão automática dos acentos)
\usepackage{lastpage}			% Usado pela Ficha catalográfica
\usepackage{indentfirst}		% Indenta o primeiro parágrafo de cada seção.
\usepackage{color}				% Controle das cores
\usepackage{graphicx}			% Inclusão de gráficos
\usepackage{microtype} 			% para melhorias de justificação
\usepackage{amsmath}

% CONFIGURAÇÕES DE PACOTES

% Informações de dados para CAPA e FOLHA DE ROSTO
\titulo{Restauração de Imagens de Placas de Veículos Usando Super-resolução}
\autor{David Campos Anchieta}
\local{Belém - PA}
\data{2017}
\orientador{Prof. Dr. Ronaldo de Freitas Zampolo}
\instituicao{%
  Universidade Federal do Pará
  \par
  Instituto de Tecnologia
  \par
  Faculdade de Engenharia da computação e Telecomunicações}
\tipotrabalho{Trabalho de Conclusão de Curso}
% O preambulo deve conter o tipo do trabalho, o objetivo, 
% o nome da instituição e a área de concentração 

%ISSUE: redigir preambulo
\preambulo{ PREÂMBULO }


% Configurações de aparência do PDF final

% alterando o aspecto da cor azul
\definecolor{blue}{RGB}{41,5,195}

% informações do PDF
\makeatletter
\hypersetup{
     	%pagebackref=true,
		pdftitle={\@title}, 
		pdfauthor={\@author},
    	pdfsubject={\imprimirpreambulo},
	    pdfcreator={LaTeX with abnTeX2},
		pdfkeywords={abnt}{latex}{abntex}{abntex2}{trabalho acadêmico}, 
		colorlinks=true,       		% false: boxed links; true: colored links
    	linkcolor=blue,          	% color of internal links
    	citecolor=blue,        		% color of links to bibliography
    	filecolor=magenta,      		% color of file links
		urlcolor=blue,
		bookmarksdepth=4
}
\makeatother

% Espaçamentos entre linhas e parágrafos 
% O tamanho do parágrafo é dado por:
\setlength{\parindent}{1.3cm}

% Controle do espaçamento entre um parágrafo e outro:
\setlength{\parskip}{0.2cm}  % tente também \onelineskip

% compila o indice
\makeindex

% Início do documento
\begin{document}

% Seleciona o idioma do documento (conforme pacotes do babel)
%\selectlanguage{english}
\selectlanguage{brazil}

% Retira espaço extra obsoleto entre as frases.
\frenchspacing 
% ELEMENTOS PRÉ-TEXTUAIS

% Capa
\imprimircapa

% Folha de rosto
\imprimirfolhaderosto

% Inserir folha de aprovação

% Isto é um exemplo de Folha de aprovação, elemento obrigatório da NBR
% 14724/2011 (seção 4.2.1.3). Você pode utilizar este modelo até a aprovação
% do trabalho. Após isso, substitua todo o conteúdo deste arquivo por uma
% imagem da página assinada pela banca com o comando abaixo:
%
% \includepdf{folhadeaprovacao_final.pdf}
%

\begin{folhadeaprovacao}
%ISSUE: folha de aprovação
  \begin{center}
    {\ABNTEXchapterfont\large\imprimirautor}

    \vspace*{\fill}\vspace*{\fill}
    \begin{center}
      \ABNTEXchapterfont\bfseries\Large\imprimirtitulo
    \end{center}
    \vspace*{\fill}
    
    \hspace{.45\textwidth}
    \begin{minipage}{.5\textwidth}
        \imprimirpreambulo
    \end{minipage}%
    \vspace*{\fill}
   \end{center}
        
   Trabalho aprovado. \imprimirlocal, 24 de novembro de 2012:

   \assinatura{\textbf{\imprimirorientador} \\ Orientador} 
   \assinatura{\textbf{Professor} \\ Convidado 1}
   \assinatura{\textbf{Professor} \\ Convidado 2}
   %\assinatura{\textbf{Professor} \\ Convidado 3}
   %\assinatura{\textbf{Professor} \\ Convidado 4}
      
   \begin{center}
    \vspace*{0.5cm}
    {\large\imprimirlocal}
    \par
    {\large\imprimirdata}
    \vspace*{1cm}
  \end{center}
  
\end{folhadeaprovacao}

% Dedicatória
\begin{dedicatoria}
%ISSUE: dedicatória
   \vspace*{\fill}
   \centering
   \noindent
   \textit{ DEDICATÓRIA} \vspace*{\fill}
\end{dedicatoria}

% Agradecimentos
\begin{agradecimentos}
%ISSUE: agradecimentos


\end{agradecimentos}

% Epígrafe
\begin{epigrafe}
%ISSUE: epígrafe
    \vspace*{\fill}
	\begin{flushright}
		\textit{O que as suas mãos tiverem que fazer, que o façam com toda a sua força,\\
    pois na sepultura, para onde você vai, não há atividade nem planejamento,\\
    não há conhecimento nem sabedoria.\\
Eclesiastes 9:10}
	\end{flushright}
\end{epigrafe}

% RESUMOS

% resumo em português
\setlength{\absparsep}{18pt} % ajusta o espaçamento dos parágrafos do resumo
\begin{resumo}
%ISSUE: resumo pt

\end{resumo}

% resumo em inglês
\begin{resumo}[Abstract]
%ISSUE: resumo en
 \begin{otherlanguage*}{english}
   This is the english abstract.

   \vspace{\onelineskip}
 
   \noindent 
   \textbf{Keywords}: latex. abntex. text editoration.
 \end{otherlanguage*}
\end{resumo}


% inserir lista de ilustrações
\pdfbookmark[0]{\listfigurename}{lof}
\listoffigures*
\cleardoublepage
% ---

% ---
% inserir lista de tabelas
% ---
\pdfbookmark[0]{\listtablename}{lot}
\listoftables*
\cleardoublepage
% ---

% ---
% inserir lista de abreviaturas e siglas
% ---
\begin{siglas}
  \item[SR] Super-resolução
  \item[HR] \textit{High resolution} ou alta resolução
\end{siglas}
% ---

% ---
% inserir lista de símbolos
% ---
\begin{simbolos}
%ISSUE: lista de símbolos
  \item[$ \Gamma $] Letra grega Gama
  \item[$ \lambda $] Lambda
  \item[$ \zeta $] Letra grega minúscula zeta
  \item[$ \in $] Pertence
\end{simbolos}
% ---

% ---
% inserir o sumario
% ---
\pdfbookmark[0]{\contentsname}{toc}
\tableofcontents*
\cleardoublepage
% ---



% ----------------------------------------------------------
% ELEMENTOS TEXTUAIS
% ----------------------------------------------------------
\textual

% ----------------------------------------------------------
% Introdução (exemplo de capítulo sem numeração, mas presente no Sumário)
% ----------------------------------------------------------
\chapter[Introdução]{Introdução}
\addcontentsline{toc}{chapter}{Introdução}
% ----------------------------------------------------------
%ISSUE: introdução

\chapter{Restauração de Imagens Usando Super-resolução}
\section{Introdução}
A disponibilidade de um sistema de aquisição de imagens em alta resolução é um aspecto positivo para todas as aplicações que se beneficiam do uso de imagens digitais. Imagens médicas com boa resolução facilitam o diagnóstico de doenças, por exemplo \cite{park2003super}. No entanto, gerar imagens de alta resolução pode ter um custo elevado, além de apresentar outros inconvenientes dependendo da aplicação. Por exemplo, um sistema de vigilância que gerasse imagens de alta resolução precisaria de robustez computacional e alta capacidade de armazenamento para funcionar continuamente. Sistemas de imagens médicas também enfrentam o dilema de obter imagens melhores sem aumentar a exposição do paciente à radiação \cite{yue2016image}.

Todas essas questões trouxeram a necessidade de técnicas de aprimoramento de imagens que não somente aumentassem a resolução espacial artificialmente, como ocorre em um processo de interpolação, mas que também recuperassem as componentes de alta frequência que se perdem durante o processo de subamostragem.



Entende-se por Super-resolução o processo de obtenção de uma ou mais imagens de alta resolução a partir de uma ou mais imagens de baixa resolução. Técnicas de super-resolução são estudadas desde os anos 1970 têm despertado um grande interesse de estudo nas últimas décadas. As aplicações de técnicas de SR são diversas e incluem aprimoramentos de imagens médicas, de imagens de rosto, de texto e impressões digitais\cite{nasrollahi2014super}.

As abordagens também se diversificaram ao longo dos anos de estudos. Os primeiros algorítimos desenvolvidos faziam o processamento no domínio da frequência, usando transformações de Fourier e Wavelet. No entanto, os métodos de SR no domínio da frequência tinham limitações acabaram por ser substituídos por métodos no domínio espacial.



\section{Apreciação das Técnicas de Super-resolução}


\chapter{Super Resolução Bayesiana}
\section{Modelo de Observação}

Definir um modelo de observação que relacione a imagem de alta resolução às imagens de baixa resolução observadas. O modelo de observação deste trabalho tenta simular cinco tipos de alteração aplicadas por sistemas de aquisição de imagens: rotação, deslocamento, espalhamento de ponto, subamostragem e ruído.

Pela conveniência do modelo de observação, a imagem de alta resolução de dimensões $m \times n$ é representada por um vetor $\mathbf{x} = [x_1, x_2, ... , x_{N-1}, x_N]^T$ onde $N ={} m \times n$. Ou seja, os valores dos pixels da imagem são rearranjados em um vetor de comprimento $N$.

Sabendo disso, a relação entre uma imagem de alta resolução $\mathbf{x}$ e uma imagem degradada $\mathbf{y}^{(k)}$ é resumida em (\ref{eq:degradation}). Onde $k = 1,2,...,K$; sendo $K$ o número total de quadros obtidos a partir da cena de alta resolução $\mathbf{x}$.

\begin{equation}
  \label{eq:degradation}
  \mathbf{y}^{(k)} = \mathbf{W}^{(k)}\mathbf{x} + \mathbf{n}^{(k)}
\end{equation}

O ruído, representado por $\mathbf{n}$ é um vetor de variáveis aleatórias Gaussianas independentes de média zero e variância $1/\beta$.

A matriz de sistema $\mathbf{W}$ aplica as transformações de rotação, deslocamento, subamostragem e espalhamento de ponto ao ser multiplicada pela imagem de alta resolução $\mathbf{x}$. Pelas propriedades da multiplicação de matrizes, é natural que as dimensões da matriz do sistema sejam $M \times N$; onde $M$ é o número de pixels da imagem resultante. Também se espera que $N \gg M$.

A matriz $\mathbf{W}$ depende de de três parâmetros de transformação: a largura da função de espalhamento de ponto $\gamma$ um ângulo de rotação $\theta$ e um vetor de deslocamento linear $\mathbf{s}$. Os elementos da matriz são dados por (\ref{eq:wmatrix}) com a função de espalhamento de ponto descrita em (\ref{eq:psf}).

\begin{equation}
  \label{eq:wmatrix}
  W^{(k)}_{ji} = \widetilde{W}^{(k)}_{ji} / \sum_{i'} \widetilde{W}^{(k)}_{ji}
\end{equation}

\begin{equation}
  \label{eq:psf}
  \widetilde{W}^{(k)}_{ji} = \exp \left\{- \frac{\|\mathbf{v}_i - \mathbf{u}^{(k)}_j\|^2}{\gamma^2} \right\}
\end{equation}

Sendo que o vetor $\mathbf{u}^{(k)}_j$ representa o centro da função de espalhamento de ponto e mapeia, na imagem de alta resolução, as coordenadas dos pontos que serão representados na imagem de baixa resolução. Estes pontos dependem do fator de subamostragem, do ângulo de rotação e do deslocamento em relação à imagem original.

\begin{equation}
  \mathbf{u}^{(k)}_j = \mathbf{R}^{(k)}(\mathbf{v}_j-\mathbf{\overline{v}})+\mathbf{\overline{v}}+\mathbf{s}_k
\end{equation}

\begin{equation}
  \mathbf{R}^{(k)} = 
  \begin{pmatrix}
    \cos \theta_k & \sin \theta_k \\
    - \sin \theta_k & \cos \theta_k
  \end{pmatrix}
\end{equation}

\chapter{Metodologia}
\chapter{Resultados}

\chapter{Conclusão}

% ELEMENTOS PÓS-TEXTUAIS
\postextual

% Referências bibliográficas
\bibliographystyle{plain}
\bibliography{referencias}


\end{document}
