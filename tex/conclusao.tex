\chapter{Conclusão}
\label{chap:conclusao}
% RESUMO DO QUE FOI FEITO
Este trabalho apresentou a teoria e relatou a implementação e os resultados do desenvolvimento de um método de Super-resolução Bayesiana aplicado na restauração de imagens de placas de veículos.
O método escolhido foi proposto inicialmente por Tipping e Bishop \cite{tipping2003bayesian}. 


Como fundamentação teórica, foi explicado o conceito de Super-resolução e as técnicas importantes que ajudam a compreender o método proposto.
O modelo de observação utilizado pelos principais métodos (incluindo o implementado neste trabalho) também foi descrito com detalhes.

A teoria e o modelo matemático usado no método desenvolvido por Tipping e Bishop foram descritas no Capítulo \ref{chap:srbayes}.
As principais características do método proposto são:

\begin{itemize}
	\item Obtêm uma imagem de alta resolução a partir de um conjunto de imagens de baixa resolução.
	\item Incorpora uma distribuição de probabilidade como conhecimento \emph{a priori} acerca da imagem de alta resolução.
	\item Usa um método de inferência Bayesiana para estimar os parâmetros de degradação e a imagem de alta resolução em si.
	\item O uso de um método iterativo de otimização baseado e gradiente para estimar tanto os parâmetros de degradação quanto a própria imagem.

\end{itemize}

Como explicado no Capítulo \ref{chap:resultados}, o método de Super-resolução se deu em duas etapas.
Na primeira, usou-se o método de Newton truncado para otimizar os parâmetros de degradação.
Na segunda, foi usado o mesmo método de otimização para otimizar a imagem de alta resolução a partir dos parâmetros encontrados na primeira etapa e dos dados observados.

% AVALIAÇÃO
A etapa de estimação dos parâmetros de degradação obteve resultados com precisão adequada considerando o custo em tempo computacional.
A estimação dos ângulos e dos deslocamentos foi capaz de diminuir o erro em relação aos parâmetros verdadeiros consideravelmente ao longo de poucas iterações.
No entanto, o modelo proposto não foi eficaz em estimar a largura da função de espalhamento de ponto com a mesma precisão.

Em uma análise subjetiva, a aplicação do método de Super-Resolução Bayesiana obteve resultados satisfatórios. 
Como é possível observar na Figura \ref{fig:results_compare}, o método de super-resolução obteve resultados qualitativamente melhores do que um método mais simples como uma interpolação. Isso se deve ao fato de que o método de super resolução é capaz de recuperar as componentes de alta resolução da imagem que se perdem durante o processo de captura.

% JUSTIFICATIVA/MOTIVAÇÃO
O tipo de ferramenta desenvolvida neste trabalho pode vir a ter considerável importância para a área forense, onde ela pode ser usada para identificar veículos em vídeos de câmeras de segurança.

% TRABALHOS FUTUROS
Como propostas para trabalhos futuros podemos citar:

\begin{alineas}
	\item A redução do custo computacional do método. Boa parte do custo computacional do método está na dependência de cálculos com grandes matrizes, o que impossibilita a restauração de imagens de maiores dimensões.
	\item A incorporação de um método de visão computacional para identificar os caracteres da placa na imagem restaurada.
\end{alineas}

% ISSUE: Falar de pelo menos mais uma proposta de trabalho futuro.
