\chapter{Conclusão}
\label{chap:conclusao}
% RESUMO DO QUE FOI FEITO
Este trabalho apresentou a teoria e relatou a implementação e os resultados do desenvolvimento de um método de Super-Resolução Bayesiana aplicado na restauração de imagens de placas de veículos.

As principais características do método proposto são:
\begin{itemize}
	\item Obtêm uma imagem de alta resolução a partir de um conjunto de imagens de baixa resolução.
	\item Incorpora uma distribuição de probabilidade como conhecimento \emph{a priori} acerca da imagem de alta resolução.
	\item Usa um método de inferência Bayesiana para estimar os parâmetros de degradação e a imagem de alta resolução em si.
	\item O uso de um método iterativo de otimização baseado e gradiente para estimar tanto os parâmetros de degradação quanto a própria imagem.

\end{itemize}

% AVALIAÇÃO
A etapa de estimação dos parâmetros de degradação obteve resultados com precisão adequada considerando o custo em tempo computacional.
A estimação dos ângulos e dos deslocamentos foi capaz de diminuir o erro em relação aos parâmetros verdadeiros.
No entanto, o modelo proposto não foi eficaz em estimar a largura da função de espalhamento de ponto com a mesma precisão.

Em uma análise subjetiva, a aplicação do método de Super-Resolução Bayesiana obteve resultados satisfatórios. 
Como é possível observar na Figura \ref{fig:results_compare}, o método de super-resolução obteve resultados qualitativamente melhores do que um método mais simples como uma interpolação. Isso se deve ao fato de que o método de super resolução é capaz de recuperar as componentes de alta resolução da imagem que se perdem durante o processo de captura.

% TRABALHOS FUTUROS
Como propostas para trabalhos futuros podemos citar: a aplicação do método à imagens
coloridas, visto que a única mudança seria na etapa de estimação da imagem,
quando o algorítmo de otimização deve ser aplicado aos três canais da imagem;
a redução do custo computacional do algoritmo, sobretudo a etapa de estimação da imagem
que depende de matrizes com mais de 15000 elementos.

% ISSUE: Falar de pelo menos mais uma proposta de trabalho futuro.
