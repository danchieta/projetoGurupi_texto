% ----------------------------------------------------------
% Introdução (exemplo de capítulo sem numeração, mas presente no Sumário)
% ----------------------------------------------------------
\chapter[Introdução]{Introdução}
% \addcontentsline{toc}{chapter}{Introdução}
% ----------------------------------------------------------
%ISSUE: introdução
%ISSUE: Corrigir a capitalização de seção
%ISSUE: Corrigir a capitalização de capítulo
%ISSUE: Corrigir a capitalização de figura 
%ISSUE: Corrigir a capitalização de tabela
%ISSUE: Corrigir a capitalização de equação

% CONTEXTUALIZAÇÃO: ONDE MEU PROBLEMA APARECE?
A disponibilidade de imagens de alta resolução é uma vantagem para a execução de qualquer trabalho que envolva a captação e interpretação de informações através de imagens digitais.
Como exemplo disso podemos citar as imagens médicas, nas quais quanto melhor a
qualidade da imagem, mais fácil é diferenciar as estruturas retratadas com o objetivo
de diagnosticar doenças.
Também há o caso dos sistemas de vigilância, nos quais quanto melhor a resolução espacial
da imagem, mais viável se torna a identificação de veículos e criminosos.

% DELIMITAÇÃO: QUE PROBLEMA ESTOU RESOLVENDO?
Hoje em dia, com as frequentes ameaças à segurança das pessoas em diversos lugares do mundo, os cidadãos sentem a necessidade de investir na própria segurança privada para proteger a si mesmos e seus patrimônios.
Por causa dessa necessidade, houve uma popularização de sistemas de vigilância monitoramento.
As câmeras de segurança já são tão onipresentes em grandes cidades que é praticamente impossível que um cidadão comum saia à rua e não tenha a sua imagem gravada por pelo menos uma delas.

No entanto, apesar da presença geral dos sistemas de vigilância, a maioria dos equipamentos utilizados é de baixa qualidade, o que se reflete nas imagens captadas por eles.
A consequência disso é que muitas vezes as câmeras de monitoramento registram cenas de crimes,
mas as imagens deterioradas nem sempre ajudam a trazer à luz informações importantes, como o rosto de um criminoso ou a placa de um carro, por exemplo.

Não obstante, mesmo com necessidade evidente de se melhorar a qualidade das imagens geradas por sistemas de vigilância,
também há o dilema constante entre aumentar a qualidade das imagens registradas e o custo de implementar manter o sistema de monitoramento funcionando.
Quanto melhor a qualidade da imagem, maior é o custo do equipamento e mais capacidade de armazenamento será necessária para guardar uma quantidade útil de todas as imagens gravadas.


% SOLUÇÕES: O QUE JÁ EXISTE QUE RESOLVE ESTE PROBLEMA?
Face ao dilema das imagens geradas por sistemas de vigilância,
a melhor solução a curto prazo seria aprimorar as imagens que estão sendo geradas pelas câmeras de monitoramento que já temos hoje.
Esse aprimoramento seria feito através de restauração de imagens digitais.

A demanda pelo desenvolvimento de técnicas de restauração de imagens ganhou força no fim da década de 1950, quando a sua principal aplicação era na área de imagens astronômicas.
Naquele contexto, era necessário restaurar as primeiras imagens recebidas de sondas e satélites artificiais.
As imagens captadas por sondas espaciais e satélites artificiais sofriam degradações próprias do ambiente onde esses objetos trabalhavam.

Após tantos anos, a restauração de imagens digitais continua tendo sua importância na astronomia, mas também é uma ferramenta para outras áreas como medicina, mídia e ciências forenses \cite{banham1997digital}.
Cada uma com suas particularidades e necessidades.

% SOLUÇÃO ESCOLHIDA: O QUE ELA É E COMO FUNCIONA?
Este trabalho descreve a implementação e teste de um método de restauração de imagens de placas de veículos usando super-resolução bayesiana.
O método utilizado foi apresentado por Michael Tipping e Christopher Bishop \cite{tipping2003bayesian}.
Em resumo, o algorítimo apresentado define um modelo de observação no qual as degradações sofridas pela imagem durante o processo de aquisição são parametrizadas.
Então são usadas inferências bayesianas para estimar os parâmetros de degradação a partir das imagens de baixa resolução, para então estimar a imagem de alta resolução a partir destes parâmetros.

% VANTAGENS: POR QUE ESTA SOLUÇÃO E NÃO AS OUTRAS?
Esta solução foi escolhida devido à adequação do algorítimo à aplicação desejada, que é a restauração de uma imagem de placa de veículo a partir de várias imagens degradadas.
A ideia principal é que o método possa ser usado para revelar os caracteres de uma placa de um veículo que tenha sido captado por uma câmera de vídeo, mas que não esteja visível devido à má qualidade da imagem.

Também vale destacar a relativa simplicidade de implementação do método escolhido, o qual também possibilita ajustes afim de reduzir seu custo computacional.

% ESTRUTURA: O QUE TEM NOS CAPÍTULOS SEGUINTES?
% ISSUE: Escrever estrutura do trabalho

Esta monografia está organizada da seguinte forma: O Capítulo \ref{chap:superRes} apresenta o conceito de Super-resolução e descreve resumidamente as principais técnicas.
O Capítulo \ref{chap:srbayes} apresenta e descreve com detalhes o método de Super Resolução Bayesiana, o qual é objeto de estudo deste trabalho.
O Capítulo \ref{chap:resultados} descreve o contexto no qual o método foi testado e apresenta os resultados das simulações.
As considerações finais acerca do trabalho são feitas no Capítulo \ref{chap:conclusao}.

