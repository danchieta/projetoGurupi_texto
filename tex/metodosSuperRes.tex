\section{Apreciação dos Métodos de Super-resolução} 
\subsection{Domínio da Frequência}

Os primeiros algorítimos de super-resolução desenvolvidos faziam o processamento das imagens no domínio da frequência.
Esse tipo de abordagem tinha a vantagem da simplicidade em sua teoria.
No entanto, as técnicas de super-resolução no domínio da frequência tinham limitações com respeito ao tipo de degradação sofrida pela imagem além da dificuldade de aplicar o conhecimento a priori da imagem de alta resolução no domínio espacial para regularização.
Por estes e outros motivos que a maioria dos algorítimos de super-resolução desenvolvidos até hoje faz o processamento das imagens no domínio espacial \cite{park2003super}.

Os trabalhos de Gerchberg \cite{Gerchberg1974} e Santis e Gori \cite{de1975iterative} foram os primeiros a introduzir técnicas de super-resolução no domínio da frequência.
Sendo que o primeiro método a ter múltiplas imagens como entrada foi desenvolvido por Tsai e Huang em 1994 \cite{nasrollahi2014super}.

% ISSUE: Completar seção domínio da frequência.


\subsection{Domínio Espacial}
A maioria dos algorítmos de super-resolução desenvolvidos faz o processamento de imagem no domínio espacial.
Os tipos de abordagens usadas são bastante diversos, sendo que a mais frequente é a probabilística, na qual o método usado neste trabalho está incluído \cite{nasrollahi2014super}.

\subsubsection{Iterative Back Projections}
% ISSUE: Tentar mudar a formatação de subsubsection

O método \emph{Iterative Back Projection} ou retroprojeção iterativa (tradução do autor) foi proposto pela primeira vez em \cite{irani1991improv}.
Trata-se de um método iterativo que visa minimizar o erro a cada execução.

% ISSUE: Inserir IBP na lista de siglas.

A cada interação do método IBP, uma imagem de alta resolução é estimada.
Então é usado o modelo de observação para gerar um conjunto simulado de imagens de baixa resolução.
Esse conjunto de imagens simuladas é comparado às imagens de baixa resolução obtidas através do sistema de aquisição.
A comparação se dá através do cálculo da diferença (erro) entre os dois conjuntos de imagem.
O método iterativo busca minimizar esse erro afim de recuperar a imagem de alta resolução \cite{park2003super,reis2014metodo}.

% SUGGESTION: Inserir figura do esquema de IBP.
% SUGGESTION: Inserir equaçõ do algoritmo de IBP.

\subsubsection{Métodos de Aprendizado}

\subsubsection{Métodos Probabilísticos}



